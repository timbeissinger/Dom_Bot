\documentclass{pnastwo}


\begin{document}

\title{Demography and selection during the maize domestication
  bottleneck}
\author{Timothy M. Beissinger\affil{1}{University of
    California,Davis}, Many many others for certain, \and Jeffrey Ross-Ibrarra\affil{1}{} }

\significancetext{This work is insignificant ;-)}

\maketitle

\begin{article}

\begin{abstract}
This is the abstract. It should probably be somewhere around 200 words. You should definitely check to be sure.
\end{abstract}

\dropcap{T}his is the beginning of the article. Notice the
dropcap... that is a neat feature that PNAS likes, I think it looks
pretty neat too!

This is an additional paragraph. The text will be repetitive for the
moment, as I am testing things out!!This is an additional paragraph. The text will be repetitive for the
moment, as I am testing things out!!This is an additional paragraph. The text will be repetitive for the
moment, as I am testing things out!!
This is an additional paragraph. The text will be repetitive for the
moment, as I am testing things out!!
This is an additional paragraph. The text will be repetitive for the
moment, as I am testing things out!!
This is an additional paragraph. The text will be repetitive for the
moment, as I am testing things out!!
This is an additional paragraph. The text will be repetitive for the
moment, as I am testing things out!!
This is an additional paragraph. The text will be repetitive for the
moment, as I am testing things out!!
This is an additional paragraph. The text will be repetitive for the
moment, as I am testing things out!!
This is an additional paragraph. The text will be repetitive for the
moment, as I am testing things out!!
This is an additional paragraph. The text will be repetitive for the
moment, as I am testing things out!!
This is an additional paragraph. The text will be repetitive for the
moment, as I am testing things out!!

\section{Results}
\subsection{Results 1}
This is the first part of the results section. This is the first part
of the results section. This is the first part of the results
section. This is the first part of the results section. This is the
first part of the results section. This is the first part of the
results section. This is the first part of the results section. This
is the first part of the results section. This is the first part of
the results section. This is the first part of the results
section. This is the first part of the results section. This is the
first part of the results section. This is the first part of the
results section. This is the first part of the results section. This
is the first part of the results section. This is the first part of
the results section. This is the first part of the results
section. This is the first part of the results section. 
We will certainly cite dadi \cite{dadi}, this is a test.

\subsection{Results 2}
This is the second part of the results section. This is the second part of the results section. This is the second part of the results section. This is the second part of the results section. This is the second part of the results section. This is the second part of the results section. This is the second part of the results section. This is the second part of the results section. This is the second part of the results section. This is the second part of the results section. This is the second part of the results section. This is the second part of the results section. This is the second part of the results section. This is the second part of the results section. This is the second part of the results section. This is the second part of the results section. This is the second part of the results section. This is the second part of the results section. This is the second part of the results section. This is the second part of the results section. This is the second part of the results section. This is the second part of the results section. This is the second part of the results section. 

\section{Discussion}
\subsection{Discussion 1}
This is the first part of the discussion section. This is the first
part of the discussion section. This is the first part of the
discussion section. This is the first part of the discussion
section. This is the first part of the discussion section. This is the
first part of the discussion section. This is the first part of the
discussion section. This is the first part of the discussion
section. This is the first part of the discussion section. This is the
first part of the discussion section. This is the first part of the
discussion section. This is the first part of the discussion
section. This is the first part of the discussion section. This is the
first part of the discussion section. This is the first part of the
discussion section. This is the first part of the discussion section. 

\subsection{Discussion 2}
This is the second part of the discussion section. This is the second
part of the discussion section. This is the second part of the
discussion section. This is the second part of the discussion
section. This is the second part of the discussion section. This is
the second part of the discussion section. This is the second part of
the discussion section. This is the second part of the discussion
section. This is the second part of the discussion section. This is
the second part of the discussion section. This is the second part of
the discussion section. This is the second part of the discussion
section. This is the second part of the discussion section. This is
the second part of the discussion section. This is the second part of
the discussion section. This is the second part of the discussion
section. This is the second part of the discussion section. This is
the second part of the discussion section. This is the second part of
the discussion section. 

\begin{materials}
\subsection{Plant materials}
Accessions studied were selected from the Maize HapMap2
panel \cite{hapmap2} . Principal component analysis was employed to ensure that
closely related individuals were not included due to their potential
to bias results. Ultimately, 23 maize inbreds derived from a diverse
assortment of landraces were selected for inclusion. Thirteen teosinte
inbred lines, all members of the subspecies Z. \emph{mays}
ssp. \emph{parviglumis}, were utilized. Sequences were mapped to the
maize B73 version 3 reference genome \cite{maizeGenome}
(ftp://ftp.ensemblgenomes.org/pub/plants/release-22/fasta/zea\_mays/dna/).


\subsection{Methods}
Here you should describe the methods used. Here you should describe the methods used. Here you should describe the methods used. Here you should describe the methods used. Here you should describe the methods used. Here you should describe the methods used. Here you should describe the methods used. Here you should describe the methods used. Here you should describe the methods used. Here you should describe the methods used. Here you should describe the methods used. Here you should describe the methods used. Here you should describe the methods used. Here you should describe the methods used. Here you should describe the methods used. 
\end{materials}

\begin{acknowledgments}
Various thankyous will be in order.
\end{acknowledgments}


\begin{thebibliography}{10}
\bibitem{dadi}
R.~Gutenkunst, R.~Hernandez,S.~Williamson, and C.~Bustamante, {\em
  Inferring the joint demographic history of multiple populations from
  multidimensional SNP frequency data}, PLoS Genetics., 5:10 (2009), e1000695.

\bibitem{hapmap2}
J.~Chia, C.~Song, P.~Bradbury, D.~Costich, N.~de~Leon, and others,
\emph{Maize HapMap2 identifies extant variation from a genome in
  flux}, Nature Genetics., 44 (2012), 803-807.

\bibitem{maizeGenome}
P.~Schnable, D.~Ware, R.~Fulton, J.~Stein, F.~Wei, \emph{The B73 maize
genome:complexity, diversity, and dynamics}, Science., 326:5956
(2009), 1112-1115.

\end{thebibliography}
\end{article}
\end{document}